% nice to guard against double loading...

\ifx\zhfont\undefined\else
  \expandafter\endinput
\fi

\directlua{lua.bytecode[0]=assert(loadfile(kpse.find_file("luatypezh.lua")))}

\catcode`@=11

\ifx\zh@rm \undefined
  % 使用 Adobe 的 Source 宋体
  % https://github.com/adobe-fonts/source-han-serif
  \def\zh@rm{sourcehanserifsc-regular}
  \def\zh@tt{sourcehanserifsc-regular}
  \def\zh@bf{sourcehanserifsc-bold}
  % 使用 Adobe 的 Source 黑体
  \def\zh@it{sourcehansanssc-regular}
  \def\zh@sl{sourcehansanssc-regular}
\fi

\newfam\zhfam % 数学中的中文字体族
\def\zhmathchars#1#2{{\countdef\i=\z@\countdef\j=\tw@
  \i=#1 \j=#2
  \loop \global\Umathcode\i \z@ \zhfam \i
  \ifnum\i<\j \advance\i \@ne \repeat}}

% 在数学环境中使用中文
\zhmathchars{"4E00}{"9FFF}
\zhmathchars{"20000}{"2EBEF} % BMP 以外的字符

\ifx\selectfont\undefined\else % NFSS
  \let\zh@textfont\nullfont
  \let\zh@scriptfont\nullfont
  \let\zh@scriptscriptfont\nullfont
  \toksapp\every@math@size{\zh@math@size}
  \def\zh@math@size{{\escapechar\m@ne
    \directlua{luatypezh.mathhook(
      [[\the\textfont\z@]],
      [[\the\scriptfont\z@]],
      [[\the\scriptscriptfont\z@]])}}%
    % XXX why global?
    \global\textfont\zhfam\zh@textfont
    \global\scriptfont\zhfam\zh@scriptfont
    \global\scriptscriptfont\zhfam\zh@scriptscriptfont}
\fi

\sfcode`”0    \sfcode`’0    \sfcode`)0

\def\zhnonfrenchspacing{%
  \sfcode`。3000\sfcode`?3000\sfcode`!3000%
  \sfcode`:2000\sfcode`;1500\sfcode`,1250%
  \sfcode`、1100}
\def\zhfrenchspacing{%
  \sfcode`。\@m\sfcode`?\@m\sfcode`!\@m
  \sfcode`:\@m\sfcode`;\@m\sfcode`,\@m
  \sfcode`、\@m}
\zhnonfrenchspacing

\let\@frenchspacing\frenchspacing
\let\@nonfrenchspacing\nonfrenchspacing

% Patch \(non)frenchspacing
\expandafter\expandafter\expandafter\def
\expandafter\expandafter\expandafter\frenchspacing
\expandafter\expandafter\expandafter
{\expandafter\frenchspacing\zhfrenchspacing}

\expandafter\expandafter\expandafter\def
\expandafter\expandafter\expandafter\nonfrenchspacing
\expandafter\expandafter\expandafter
{\expandafter\nonfrenchspacing\zhnonfrenchspacing}

% assign a `kindcode' to cjk characters
% 1: left punct (
% 2: right punct )
% 3: ordinary cjk
\def\zhsetkind#1#2{\begingroup
  \def\next##1{\ifx##1\relax\let\next\endgroup
    \else\zhs@tkind{`##1}{#1}\fi\next}%
  \next#2\relax}
\def\zhs@tkind#1#2{%
  \directlua{luatypezh.setkind(\number#1,#2)}}

\def\zh@ttwd{%
  % XXX em-size of lmtt10 (and maybe others) is broken
  % try to calculate the real size
  \setbox\z@\hbox{X}}

\def\zh@font#1#2#3{\relax
  \directlua{luatypezh.setup(\fontid#1,\fontid#2,#3)}}
\def\zhfont#1#2{\zh@font{#1}{#2}{false}}
\def\zhttfont#1#2{{#1\zh@ttwd\zh@font{#1}{#2}{true}}}

% Plain TeX
\ifx\tenrm\undefined\else
  \def\zh@setup@{% XXX 每次启动时载入字体比把字体 \dump 在 .fmt 里更快
    \font\tenitzh "\zh@it" at 10\p@
    \font\tenslzh "\zh@sl" at 10\p@
    \font\tenbfzh "\zh@bf" at 10\p@
    \font\tenttzh "\zh@tt" at 10\p@
    \font\tenrmzh "\zh@rm" at 10\p@
    \font\sevenrmzh "\zh@rm" at 7\p@
    \font\fivermzh "\zh@rm" at 5\p@
    % 数学字体使用 10pt 宋体
    % 如需临时使用其他字体,请使用 \hbox{...}
    \textfont\zhfam \tenrmzh
    \scriptfont\zhfam \sevenrmzh
    \scriptscriptfont\zhfam \fivermzh
    % 设置字体替代
    \zhfont\tenrm\tenrmzh
    \zhfont\sevenrm\sevenrmzh
    \zhfont\fiverm\fivermzh
    \zhfont\tenit\tenitzh
    \zhfont\tensl\tenslzh
    \zhfont\tenbf\tenbfzh
    \zhttfont\tentt\tenttzh
  }
  % 把 \beginsection 修改为首段也缩进
  \let\@beginsection\beginsection
  {\def\strip#1\noindent{#1}
  \expandafter\expandafter\expandafter \outer
  \expandafter\expandafter\expandafter \gdef
  \expandafter\expandafter\expandafter \beginsection
  \expandafter\expandafter\expandafter #%
  \expandafter\expandafter\expandafter 1%
  \expandafter\expandafter\expandafter \par
  \expandafter\expandafter\expandafter {%
  \expandafter\strip\beginsection#1\par}}
\fi

% LaTeX
\ifx\selectfont\undefined\else
  \def\zh@nfss#1#2#3{\relax
    \directlua{luatypezh.nfss("#1","#2",#3)}}
  \def\zhnfss#1#2{\zh@nfss{#1}{#2}{false}}
  \def\zhnftt#1#2{\zh@nfss{#1}{#2}{true}}
  \def\zh@nfssupdate{{\zh@ttwd\escapechar\m@ne
    \directlua{luatypezh.nfsshook([[\font@name]])}}}
  % patch \selectfont
  \begingroup
    \expandafter\let\expandafter\next\csname selectfont \endcsname
    \expandafter\long\expandafter\def\expandafter\next
      \expandafter{\next\zh@nfssupdate}
    \expandafter\global\expandafter\let\csname selectfont \endcsname\next
  \endgroup

  \def\zh@setup@{% poor NFSS hack
    \zhnfss{.*/.*/.*/.*/}\zh@rm
    \zhnftt{.*/.*tt/.*/.*/}\zh@tt
    \zhnftt{.*/.*tt/bx/.*/}\zh@bf
    \zhnfss{.*/.*/bx/.*/}\zh@bf
    \zhnfss{.*/.*/.*/it/}\zh@it
    \zhnfss{.*/.*/.*/sl/}\zh@sl
  }
\fi

\def\zh@setup{% 由 Lua 记录的状态必须在 \everyjob 中设置
  \directlua{luatypezh=lua.bytecode[0]()}%
  %
  % 以下标点符号将被压缩左边界
  \zhsetkind1{〈《「『【〔({‘“}% 左括号
  % 以下标点符号将被压缩右边界
  \zhsetkind2{〉》」』】〕)}’”}% 右括号
  \zhsetkind2{、。,.?!:;}%
  % 以下标点符号被当作普通字符处理
  \zhsetkind3{—…·¥}%
  %
  \zh@setup@
}

\ifx\INITEX+
  \let\INITEX\undefined
  \everyjob{\zh@setup}
\else
  \zh@setup
\fi

\catcode`@=12
